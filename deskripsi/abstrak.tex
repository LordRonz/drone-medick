% Mengubah keterangan `Abstract` ke bahasa indonesia.
% Hapus bagian ini untuk mengembalikan ke format awal.
\renewcommand\abstractname{Abstrak}

\begin{abstract}

  % Ubah paragraf berikut sesuai dengan abstrak dari penelitian.
  Drone atau yang biasa juga disebut sebagai pesawat nirawak merupakan pesawat yang tidak memiliki pilot manusia, kru, maupun penumpang yang membuat pesawat ini sepenuhnya independen. Teknologi drone menyediakan keuntungan yang sangat melimpah dan memberi kesempatan yang luas untuk banyak bidang penelitian. Drone dapat melakukan hal-hal seperti halnya surveying, humanitarian work, manajemen resiko bencana, riset dan juga transportasi. Penggunaan drone yang kerap terjadi pada bidang kesehatan atau medis biasanya berupa penyaluran alat-alat paket pertolongan pertama, obat-obatan, penyaluran vaksin, darah, dan kebutuhan kesehatan lainnya yang ditujukan ke daerah terpencil. Jaringan IoD ini dapat menjadi target dari beberapa ancaman keamanan dan privasi yang berbahaya dan jahat. Baik drone-drone maupun entiti IoD lain mungkin saja dibajak untuk tujuan serangan siber, data breaches, atau pencurian data dengan menggunakan payload. Penggunaan produk kesehatan  dari darah yang cepat, termasuk sel darah merah yang dikemas (PRBC), plasma, dan trombosit, telah terbukti menyelamatkan nyawa pada pasien yang menderita trauma perdarahan. Jadi, bahkan dengan perluasan pusat trauma dalam 2 dekade terakhir, banyak orang Amerika masih memiliki akses terbatas dan berpotensi mendapat manfaat dari tingkat perawatan lokal yang lebih tinggi. Drone digunakan untuk pengawasan lokasi bencana, area dengan bahaya biologis dan kimia, dan pelacakan lokasi penyebaran penyakit atau pandemi. Telah ditunjukkan bahwa drone dapat mengumpulkan informasi tentang jumlah pasien yang membutuhkan perawatan dan triase di lingkungan berisiko tinggi. Terlepas dari masalah keamanan dan privasi jaringan IoD, drone, yang merupakan komponen utama jaringan IoD, rentan terhadap ancaman fisik yang memengaruhi keselamatan mereka, yang menghambat pencapaian misi. Model ancaman adalah prosedur di mana potensi kerentanan atau serangan diidentifikasi dan mitigasinya dapat ditentukan. Model menggambarkan sifat penyerang, vektor serangan, area jaringan yang mudah diserang, dan tindakan pengendalian yang harus diambil. Drone adalah perangkat kendala sumber daya yang ditandai dengan daya komputasi yang rendah, kapasitas memori yang rendah, dan konsumsi energi yang rendah. Dengan demikian, teknik mitigasi tradisional untuk mengekang serangan yang berlaku untuk arsitektur pesawat serupa lainnya mungkin tidak diterapkan di internet drone (IoD). Untuk memitigasi serangan man-in-the-middle, eavesdropping, dan wormhole di internet drone (IoD), altawy telah menyarankan penggunaan protokol enkripsi kriptografi. Dalam skema yang diusulkan oleh pigatto, modul pemeriksaan kesehatan terpusat yang menjamin operasi drone yang lebih aman di IoD dalam mencoba mengurangi serangan analisis lalu lintas. 

\end{abstract}

% Mengubah keterangan `Index terms` ke bahasa indonesia.
% Hapus bagian ini untuk mengembalikan ke format awal.
\renewcommand\IEEEkeywordsname{Kata kunci}

\begin{IEEEkeywords}

  % Ubah kata-kata berikut sesuai dengan kata kunci dari penelitian.
  Drone Medis, Keamanan, Teknologi, Privasi, \emph{IoD}, Mitigasi.

\end{IEEEkeywords}
