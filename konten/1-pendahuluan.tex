% Ubah judul dan label berikut sesuai dengan yang diinginkan.
\section{Pendahuluan}
\label{sec:pendahuluan}

% Ubah paragraf-paragraf pada bagian ini sesuai dengan yang diinginkan.

Drone atau yang biasa juga disebut sebagai pesawat nirawak merupakan pesawat yang tidak memiliki pilot manusia, kru, maupun penumpang yang membuat pesawat ini sepenuhnya independen. Pesawat nirawak yang juga disebut dalam bahasa inggris sebagai \emph{UAV} atau \emph{Unmanned  Aerial Vehicle} merupakan sebuah komponen daripada \emph{UAS} atau \emph{Unmanned Aerial System} yang menyertakan kontroller yang berbasis di daratan dan sistem komunikasi dengan \emph{UAV}-nya itu sendiri \citep{HU2018162}. Terbangnya pesawat nirawak dapat dikendalikan oleh manusia sebagaimana \emph{remote-piloted aircraft(RPA)}, atau dengan beberapa teknik autonomous, seperti bantuan autopilot, hingga pesawat yang benar-benar autonomous sehingga sama sekali tidak ada intervensi dari manusia \citep{caryleslie2011}. 

UAV ini awalnya dikembangkan selama abad ke-20 yang diutamakan untuk melakukan misi militer yang dianggap terlalu kotor atau berbahaya untuk dilakukan oleh manusia, dan pada abad ke-21 drone ini sudah menjadi hal krusial yang harus ada pada kemiliteran. Seiring dengan berjalannya waktu dan harga dari drone atau UAV ini menurun maka penggunaannya merambah ke hal-hal yang tidak berbau militer \citep{tice2009}\citep{9423979}. Hal ini termasuk dengan fotografi udara, pengantaran produk atau barang, agrikultur, surveillance, inspeksi infrastruktur, sains\citep{drones4010005}\citep{w10050655}\citep{w10030264}\citep{w11030604}, penyelundupan\citep{dronesmuggle}, dan balapan drone. 

Teknologi drone menyediakan keuntungan yang sangat melimpah dan memberi kesempatan yang luas untuk banyak bidang penelitian. Drone dapat melakukan hal-hal seperti halnya surveying, humanitarian work, manajemen resiko bencana, riset dan juga transportasi\citep{AYAMGA2021120677}. Dalam bidang agrikultur, drone dapat melakukan imagery real-time dan sensor data dari lahan pertanian yang luas yang tidak dapat diakses dengan cepat menggunakan kaki ataupun kendaraan\citep{AYAMGA2021120677}.

Perkembangan drone berakar dalam dalam sejarah militer. Angkatan Laut AS bersama dengan tim peneliti ulang Inggris di Ordnance College of Woolwich pertama kali bereksperimen dengan torpedo udara dalam upaya memerangi U-boat Jerman dalam Perang Dunia I (Perang Dunia I). Upaya ini memicu penyelidikan terhadap pesawat tanpa pilot. Dari tahun 1922 hingga 1925, Angkatan Laut menguji sistem kontrol radio pada Pesawat N-9. Pada tahun 1924, penerbangan radio-kontrol pertama yang berhasil dilakukan dari lepas landas hingga mendarat \citep{rife2006sound}.

Global Positioning System atau GPS dan juga aplikasi untuk smartphone dan tablet dan meningkatkan kualitas prediksi durasi penerbangan, lebih reliable, dan kemudahan penggunaan serta kemampuan untuk memanfaatkan kamera yang lebih baik dan juga sensor-sensor lain yang dibutuhkan untuk mengaplikasikan drone pada agrikultur dan sumber daya alam\citep{AYAMGA2021120677}. Penggunaan drone menjadi sangat intim dengan beberapa sektor yang dikembangkan dengan ekonomi berkembang. Jika kita kehilangan drone itu maka dapat mengakibatkan implikasi yang berakibat merusak.

Penggunaan drone yang kerap terjadi pada bidang kesehatan atau medis biasanya berupa penyaluran alat-alat paket pertolongan pertama, obat-obatan, penyaluran vaksin, darah, dan kebutuhan kesehatan lainnya yang ditujukan ke daerah terpencil. Hal ini dapat memberikan transportasi test sample yang aman dari penyakit dengan tingkat penularan yang tinggi karena tidak memerlukan manusia untuk langsung terjun di lapangan, dan juga dapat memberikan akses cepat kepada external defibrilator otomatis untuk pasien yang menderita \emph{cardiac arrest} untuk menyelamatkan nyawanya\citep{AYAMGA2021120677} dan selama masa gawat darurat kesehatan. Di era pandemi COVID-19 ini drone dapat mengantar atau menyalurkan \emph{Personal Protective Equipments} (PPE), alat test, vaksin, pengobatan, dan sample dari laboratorium. 

Drone dapat membantu untuk melakukan inspeksi social distancing yang mudah secara otomatis\citep{Ramadass2020ApplyingDL}. Sebagai teknologi baru, drone dapat menyediakan solusi dari konteks pada keadaan ekstrim darurat, topografi yang sulit, dan infrastruktur transportasi. Pengadopsian drone untuk melakukan pengantaran atau penyaluran benda-benda yang krusial dan obat-obatan yang penting untuk keselamatan kepada seluruh masyarakat dengan keadaan ekonomi apapun dapat mewujudkan kesetaraan kesehatan universal\citep{mccall2019}. Keuntungan yang didapatkan dari drone pada sektor transportasi adalah keuntungan logistik dan juga transportasi penumpang\citep{KELLERMANN2020100088}.

Ekspansi dari penggunaan drone yang pada awalnya hanya digunakan untuk keperluan militer hingga menjadi keperluan sipil juga membuat adanya urgensi untuk melakukan pengembangan teknologi dari drone itu sendiri menjadi lebih baik dan memanfaatkan segala potensi drone yang ada demi masyarakat yang lebih baik. \cite[Greenwood (2016)]{Greenwood} juga mengatakan bahwasanya untuk menyadari potensi penuh dari sebuah teknologi drone ini, peraturan yang meregulasi penggunaan drone ini sangat diperlukan sembari tetap mengutamakan keamanan masyarakat dan hak-hak privasinya juga. Penyalahgunaan seperti contohnya terorisme, privasi dan penggunaan militer merupakan resiko yang dikhawatirkan terjadi pada penggunaan drone\citep{risa2015}.

Meskipun terdapat isu-isu tersebut, \cite[Sylvester (2018)]{sylvester2018agriculture} mengatakan bahwa teknologi drone ini dapat memberikan pekerjaan untuk para pemuda yang dapat menggunakan drone untuk menyediakan layanan untuk petani-petani di daerah pedesaan. Berlawanan dengan latar belakang ini, review ini berusaha untuk memperlihatkan pengembangan saat ini daripada penggunaan drone yang ada di sektor agrikultur, bidang kesehatan, dan juga kemiliteran. Untuk menggapai target ini, kita awalnya harus diperlihatkan latar belakang teknikal secara singkat dari drone ini dan pada sektor ini dan juga reviewnya mengambil dari pendekatan analisis SWOT.

\emph{Internet of Drones (IoD)} meniru akronim dari IoT yang menempatkan "Drones" untuk menggantikan "Things". Maka, IoD ini memiliki beberapa kesamaan dengan IoT atau Internet of Things ini. \citet{gharibi2016internet} mendefinisikan IoD sebagai arsitektur jaringan kontrol yang berlapis yang dapat membantu untuk mengkoordinasikan drone-drone \citep{choudhary2018internet}. Paradigma jaringan IoD ini dapat diaplikasikan dalam operasi Search and Rescue, monitoring angkatan militer, inspeksi industrial, monitoring infrastruktur, sistem pengantaran barang\citep{times2020food}, agrikultur \citep{yazdinejad2020enabling} \citep{boursianis2020internet}, mapping supply chain, manajemen bencana \citep{magistretti2019unveiling} \citep{paddeu2019new}, dan lain sebagainya. Terdapat ekspektasi kuat yang adalah IoD dapat memiliki peran yang signifikan pada smart city di masa depan \citep{vattapparamban2016drones}. Layanan publik tingkat lanjut sekarang biasanya sekarang dapat mengadakan operasi risiko kritikal alami maupun buatan manusia dengan menggunakan IoD \citep{polka2017use} \citep{kharchenko2018cybersecurity}. 

Akan tetapi, jaringan IoD ini dapat menjadi target dari beberapa ancaman keamanan dan privasi yang berbahaya dan jahat. Baik drone-drone maupun entiti IoD lain mungkin saja dibajak untuk tujuan serangan siber, data breaches, atau pencurian data dengan menggunakan payload. Berdasarkan dari author pada \citep{thiobane2015cybersecurity}, sebuah drone phantom DJI ketika dibajak dapat dijual belikan pada situs ebay dengan harga sebesar 1,000 US dollar. Authornya juga menyatakan bahwa sebuah kamera drone yang digunakan pada industri film dapat dihargai sampai dengan harga 20,000 US dollar, dan sebuah detektor cahaya dan range (LIDAR) sensor dapat diberi harga hingga mencapai 50,000 US dollar. Lebih lagi, ketika sebuah drone yang membawa data yang berharga dibajak, kerugiannya dapat mencapai ribuan US dollar. Kerugian yang jauh lebih besar dapat terjadi ketika drone yang diserang merupakan drone untuk militer. Dampaknya bukan saja hanya membocorkan data berharga atau rahasia ataupun kerusakan fisik dari drone bersangkutan namun juga drone yang dibajak dapat dijadikan sebagai sebuah senjata oleh orang yang membajak \citep{thiobane2015cybersecurity}. Komunikasi yang terjadi antara drone yang ada di dalam jaringan IoD adalah melalui internet yang tidak aman (umumnya jaringan nirkabel atau wireless maupun WiFi) dan menggunakan sinyal navigasi (contohnya global positioning system (GPS)) \citep{rodrigues2019authentication}. 

Hal ini dapat mempengaruhi aspek privasi dan keamanan pada drone secara signifikan. Hacker yang tidak bertanggungjawab dapat dengan mudah mengakses konfigurasi dari drone dan membajaknya dengan menggunakan aplikasi open-source untuk membajak drone (contohnya skyjack) dan secara nirkabel mendapatkan kendali dari drone yang menjadi target. Kebanyakan ancaman privasi dan keamanan yang ada pada drone sipil terjadi karena kesalahan pada desainnya. Kebanyakan drone dirancang tanpa perlindungan internet security dan mekanisme autentikasi \citep{rahman2017smart}. Meskipun secara tingkat kesulitan lebih sulit untuk membajak drone pada militer dikarenakan infrastruktur keamanannya yang lebih tinggi apabila dibandingkan dengan drone sipil, seorang hacker yang handal dapat menggunakan teknik yang lebih canggih. Sebuah contoh dari hal ini adalah CIA RQ-170 Sentinel US spy drone dibajak oleh hacker yang berasal dari Iran pada desember 2011 \citep{mohan2016cybersecurity}.

Sudah banyak teknik sekuriti, keamanan dan privasi yang dikembangkan oleh peneliti-peneliti untuk mendapatkan jaminan dari keamanan dari jaringan \emph{IoD} atau \emph{Internet of Drones} ini. Teknik yang ditujukan kepada memitigasi atau mencegah masalah yang mempengaruhi lokalisasi keamanan dari drone atau kebutuhab sekuriti yang diasosiasikan dengan jaringan IoD. Serangan lokalisasi error mengganggu kemampuan positioning daripada drone yang terdapat di dalam jaringan IoD sehingga dapat menyebabkan kerusakan yang besar pada performa keseluruhan dari jaringan IoD. Lebih lagi, kebutuhan privasi dan keamanan merupakan tujuan yang menentukan kesanggupan dan fungsi dari jaringan IoD yang didapatkan dari memitigasi ancaman keamanan dan privasi tertentu \citep{yahuza2020systematic}. Kebutuhan akan keamanan dan privasi dari jaringan IoD ini termasuk dengan integrity, availability, confidentiality, dan privacy preservation.

Malaysia, negara berkembang pesat di Asia, berkomitmen untuk meningkatkan kesehatan ibu melalui berbagai inisiatif strategis seperti pengenalan Penyelidikan Rahasia tentang Kematian Ibu (CEMD) dan pengembangan layanan kesehatan pedesaan \citep{achanna2018maternal}. Salah satu parameter penting kesehatan ibu adalah Angka Kematian Ibu (AKI). Keberhasilan luar biasa dicapai pada awalnya, namun, MMR kemudian mendatar dan Malaysia gagal memenuhi target WHO MDG-5 untuk mengurangi AKI sebesar 75\% pada tahun 2015. Drone telah digunakan selama bencana di Haiti, Amerika Serikat, Kanada, Karibia, dan Nepal dalam mengirimkan pasokan medis. Itu juga digunakan untuk mengirimkan Automated External Defibrillator (AED) kepada korban serangan jantung di Belanda, dan alat tes HIV di Malawi, Afrika. Contoh-contoh ini menunjukkan penggunaan drone yang meluas sebagai produk medis masa depan. transportasi di seluruh dunia. Meskipun demikian, masih banyak yang harus dilakukan oleh para peneliti untuk memberikan bukti manfaat dan meningkatkan penggunaan teknologi ini secara maksimal termasuk penerapan drone untuk hasil kesehatan ibu.

Terlepas dari minat yang besar, saat ini tidak ada tinjauan sistematis tentang penggunaan drone atau UAV dalam perawatan kesehatan ibu. Oleh karena itu kami berusaha untuk mengisi kesenjangan pengetahuan ini dengan memulai tinjauan sistematis tentang penggunaan drone dalam meningkatkan kesehatan ibu, terutama selama kedaruratan kebidanan seperti PPH. Tujuan dari makalah ini juga adalah untuk menyoroti kerangka potensial penelitian masa depan dalam pengembangan drone khusus kesehatan ibu.

Pembahasan pada paper ini dimulai dengan presentasi mengenai penelitian lain (Bagian \ref{sec:penelitianterkait}).
Kemudian dilanjutkan dengan penelitian terkait (Bagian \ref{sec:penelitianterkait}).
Setelah itu dilanjutkan dengan pencarian literatur (Bagian \ref{sec:pencarianliteratur}).
Lalu dilanjutkan dengan bab pengaplikasian (Bagian \ref{sec:pengaplikasian}).
Setelah itu dibahas bab public health (Bagian \ref{sec:publichealth})
Dengan didasari oleh bab-bab sebelumnya, maka dilanjutkan dengan bagian keamanan drone (Bagian \ref{sec:keamanandrone}).
Setelah masalah keamanan dibahas, maka diusulkan solusi keamanannya (Bagian \ref{sec:solusikeamanan}).
Terakhir, didapatkan kesimpulan dari penelitian yang telah dilakukan (Bagian \ref{sec:kesimpulan}).
