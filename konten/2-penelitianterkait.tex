% Ubah judul dan label berikut sesuai dengan yang diinginkan.
\section{Penelitian Terkait}
\label{sec:penelitianterkait}

% Ubah paragraf-paragraf pada bagian ini sesuai dengan yang diinginkan.
Aspek keamanan dalam drone sipil sudah direview pada \citep{altawy2016security}. Kemudian, beberapa serangan keamanan pada drone sudah dianalisis pada \citep{he2017drone}, \citep{yampolskiy2013taxonomy}, \citep{sedjelmaci2018cyber}, \citep{humphreys2012statement}, dan \citep{shepard2012evaluation}. Metode-metode untuk melakukan deteksi drone dianalisis pada \citep{guvencc2017detection}, \citep{sturdivant2017systems}, \citep{shi2018anti}, dan \citep{nassi2019}. Akan tetapi, keterbatasan utama dari penelitian-penelitian sebelumnya adalah kurangnya analisis yang lebih mendalam dari vulnerabilities dari dronenya itu sendiri serta kurangnya analisis attack life cycle. Terlebih lagi, hanya satu aspek dari keamanan drone yang dianalisis, yakni penyerangan pada drone. Teknik penanggulangan yang saat ini telah ada perlu dianalisis, dan teknik baru perlu diusulkan untuk mengatasi kekurangan dari solusi keamanan pada drone yang ada pada saat ini. 

Menurut \citet{yao2019qos}, arsitektur IoD pertama dirancang oleh \citet{gharibi2016internet}. Arsitektur tersebut terdiri dari lima layer konseptual (air space layer, node-to-node layer, end-to-end layer, services layer, dan application layer). Setiap layer dapat mengakses layanan yang sudah diberikan oleh layer dibawah layer tersebut. \citet{lin2018security} melakukan penelitian lebih lanjut tentang arsitektur milik gharibi dan menunjukkan kelebihan dan kelemahan dari arsitektur tersebut. Arsitektur tersebut dapat memberikan pencegahan terhadap tabrakan drone saat berada di udara. Kemudian juga dapat memberikan kontrol lebih dimana tempat yang dapat dicapai dan tidak oleh drone. Akan tetapi, terdapat kelemahan yang mana adalah kurangnya penjaluran efektif, kontrol penyumbatan, dan tantangan akan keamanan dan privasi (penyaluran data yang tidak aman). Author mengusulkan solusi yang kira-kira dapat mengatasi permasalahan yang sudah dianalisis yang nantinya akan sesuai dengan arsitektur jaringan IoD. 

Terlebih lagi, author di \citep{aggarwal2019new} mengusulkan penambahan teknologi blockchain pada layer IoD untuk membuat IoD semakin rahasia, aman, dan \emph{tamperproof}. \citet{qureshi2016dronemap} mengusulkan sebuah arsitektur IoD berbasis cloud untuk menyediakan virtualisasi akses pada drone melalui cloud dan mengunggah komputasi yang berat ke cloud dengan batasan resource yang terbatas. Arsitektur ini tersusun atas tiga buah layer. Layer yang pertama adalah drone layer yang merepresentasikan susunan resource ataupun layanan yang diberikan kepada end-users. Pada layer kedua, layer ini disebut sebagai layer layanan cloud. Layer kedua terdiri atas tiga komponen (komponen penyimpanan yang berguna untuk menyimpan data yang didapatkan dari drone, komponen komputasi, dan komponen interface atau antarmuka). Akhirnya, layer ketiga disebut sebagai layer klien. Layer ini memiliki antarmuka dari kedua layer sebelumnya yaitu layer drone dan layer cloud. 

\citet{zhang2020multidomain} mengusulkan sebuah centralized multi-layered virtual network mapping architecture. Arsitektur tersebut menggunakan virtualisasi dari fungsi jaringan yang menggunakan progres teknologi arsitektur dari IoD milik peneliti-peneliti terdahulu.
