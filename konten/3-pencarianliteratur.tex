\section{Pencarian Literatur}
\label{sec:pencarianliteratur}

Pencarian literatur secara sistematis dilakukan untuk menilai karya ilmiah yang melibatkan aplikasi medis drone saat ini. Layanan Penemuan EBSCO (Elton B. Stephens Company) digunakan sebagai mesin pencari. Pencarian lanjutan dilakukan untuk mengidentifikasi sumber yang mengandung frasa "drone," "UAV," "kendaraan udara tak berawak," "UAS," dan "sistem udara tak berawak" sebagai istilah subjek. Sumber disusun secara kronologis, dan judulnya disaring untuk relevansi dan dipilih jika dianggap dapat diterapkan. Jenis sumber termasuk majalah, jurnal akademik, artikel berita, publikasi perdagangan, dan sumber daya elektronik. Semua sumber yang diterbitkan dalam bahasa Inggris hingga April 2017 disertakan. Hasil pencarian duplikat tidak diikutkan dalam hasil.

Selanjutnya, dipilih sumber yang membahas penerapan drone di sektor sipil dan dikelompokkan ke dalam 7 kategori besar: pertanian, lingkungan dan konservasi, penegakan hukum dan lalu lintas, pendidikan, konstruksi dan industri, pelayaran komersial, dan obat-obatan. Baik sumber akademik maupun nonakademik diterima. Sumber akademik didefinisikan sebagai sumber yang diterbitkan dalam jurnal ilmiah atau prosiding dari konferensi nasional. Sumber nonakademik dimasukkan dalam upaya untuk menangkap informasi terbaru dalam pelaporan tentang teknologi howdrone yang saat ini digunakan. Sumber yang membahas aplikasi yang sama disertakan. Dari artikel-artikel ini, literatur yang relevan diekstraksi.

Pencarian tambahan digunakan untuk mengidentifikasi sumber yang mengandung istilah "drone" baik dalam istilah subjek atau judul, dan kata "obat" dalam setiap aspek teks. Tujuan dari pencarian ini adalah untuk mengisolasi sumber medis yang mungkin terlewatkan dalam pencarian awal. Paradigma yang digunakan untuk mengolah sumber dari pencarian awal diterapkan, termasuk majalah, jurnal akademik, artikel berita, publikasi perdagangan, dan sumber elektronik dalam bahasa Inggris hingga April 2017. Hasil pencarian disusun secara kronologis. Hasil pencarian duplikat atau artikel yang ditemukan dalam pencarian awal dikecualikan. Sumber yang berkaitan dengan aplikasi medis selanjutnya diindeks ke dalam 3 kategori utama: kesehatan masyarakat/bantuan bencana, telemedicine, dan transportasi medis.

Tema utama dalam kesehatan masyarakat/bantuan bencana meliputi perawatan korban massal, pengumpulan data, penyakit menular, bantuan bencana, dan pengobatan darurat. Dalam kategori telemedicine, deskripsi termasuk drone yang membantu dalam prosedur bedah di lingkungan yang keras yang disimulasikan, termasuk medan perang, dan penggunaan perangkat astelemedis drone dalam pengaturan darurat. Barang perbekalan dan transportasi medis melibatkan beberapa subkategori, antara lain pengiriman barang medis, evakuasi pasien, dan aplikasi komersial untuk infrastruktur.
