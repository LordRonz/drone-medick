\section{Pengaplikasian}
\label{sec:pengaplikasian}

Penggunaan produk kesehatan  dari darah yang cepat, termasuk sel darah merah yang dikemas (PRBC), plasma, dan trombosit, telah terbukti menyelamatkan nyawa pada pasien yang menderita trauma perdarahan \citep{berns1998blood} \citep{higgins2012red} \citep{jenkins2014implementation} \citep{riskin2009massive} \citep{mitra2010fresh} \citep{pidcoke2012ten}. Meskipun banyak rumah sakit yang memiliki akses kritis memiliki produk suplai darah yang tersedia, persediaan terkadang terbatas, dan pasokan trombosit dan plasma biasanya lebih terbatas daripada produk sel darah merah yang ada. Rumah sakit dengan akses kritis didefinisikan sebagai rumah sakit dengan 25 tempat tidur atau kurang yang terletak jaraknya setidaknya 35 mil dari rumah sakit lain melalui jalan utama atau berjarak sekitar 15 mil melalui jalan sekunder \citep{reid2014workforce}. 

Pusat trauma Tingkat III, meskipun tidak identik dengan rumah sakit dengan akses kritis, seringkali juga terletak di daerah pedesaan dan menyediakan akses penting bagi pasien trauma di daerah tersebut. Sejak awal 1990-an, jumlah pusat trauma tingkat III di Amerika Serikat telah meningkat, tetapi mereka memiliki sumber daya yang terbatas, terutama pusat-pusat di daerah pedesaan \citep{mackenzie2003national}. Selain itu, 46,7 juta orang Amerika masih tidak memiliki akses ke pusat trauma tingkat I atau II dalam waktu satu jam dari rumah mereka, dan tambahan 81,4 juta orang Amerika tidak akan, tanpa layanan helikopter, memiliki akses ke pusat trauma dalam waktu satu jam dari rumah \citep{branas2005access}. 

Jadi, bahkan dengan perluasan pusat trauma dalam 2 dekade terakhir, banyak orang Amerika masih memiliki akses terbatas dan berpotensi mendapat manfaat dari tingkat perawatan lokal yang lebih tinggi. Meskipun pusat trauma harus memiliki produk darah segera tersedia, pasokan ini tidak terbatas, dan cadangan yang lebih besar biasanya tidak tersedia. Standar perawatan saat ini merekomendasikan pengangkutan pasien yang membutuhkan transfusi ke rumah sakit yang lebih besar ketika sumber daya, termasuk produk darah, tidak tersedia atau terbatas. Ini seringkali merupakan proses yang mahal dan dapat menunda perawatan awal yang tepat. Upaya telah dilakukan untuk mengatasi masalah ini dengan mengangkut PRBC dan plasma dengan tim transportasi sebelumnya. Meskipun inovatif, perubahan tersebut tidak mengatasi biaya operasi pesawat berawak yang signifikan atau risiko bagi awak penerbangan yang bepergian di daerah terpencil. Selain itu, bencana alam dan insiden korban massal dapat terjadi di lokasi terpencil yang membutuhkan pasokan darah sementara, dan transportasi dapat menjadi penghalang yang signifikan untuk membangun stasiun-stasiun operasi maju ini.

Kemampuan rumah sakit akses kritis untuk mempertahankan inventaris produk darah diperumit oleh banyak faktor, termasuk umur simpan dan biaya. Rumah sakit mungkin memiliki berbagai jenis produk PRBC dan 3 jenis plasma yang tersedia untuk mencegah keterlambatan transfusi darurat. Meskipun masa simpan PRBC (42 hari) dan plasma (1 tahun) relatif lama, produk lain, seperti trombosit (5 hari) dan plasma yang dicairkan (5 hari), dapat terbuang sia-sia ketika permintaan rendah \citep{redcross2014}. Rumah sakit akses kritis memiliki persediaan produk darah yang terbatas dibandingkan dengan rumah sakit perawatan tersier yang besar (Tabel 1). Pada pasien dengan perdarahan hebat, transfusi masif (10 unit dalam 24 jam atau 5 unit dalam 60 menit) mungkin diperlukan, yang seringkali dapat dengan cepat menghabiskan suplai darah rumah sakit \citep{krumrei2012comparison}. Rata-rata pasien trauma yang menjalani transfusi masif membutuhkan rata-rata 22 unit PRBC dan 14 unit trombosit, lebih banyak PRBC daripada stok rumah sakit akses paling kritis \citep{holcomb2011increased}. Resusitasi transfusi masif awal juga mencakup plasma, dan rumah sakit akses kritis biasanya memiliki persediaan produk ini yang terbatas. 

Bank darah regional yang memasok rumah sakit akses kritis menyimpan cukup darah beku untuk memenuhi permintaan reguler. Selama masa permintaan tinggi atau mungkin hanya untuk 1 pasien dengan perdarahan masif, suplai darah dari rumah sakit dengan akses kritis mungkin akan habis dan memerlukan dukungan intensif dari pusat darah regional \citep{schmidt2002blood} \citep{erickson2008management}. Contohnya adalah selama gempa bumi di Bam, Iran. Peristiwa ini menyoroti inefisiensi dari proses saat ini dimana darah didistribusikan. Meskipun 108.985 unit darah disumbangkan, hanya 23\% dari unit ini yang benar-benar didistribusikan ke rumah sakit. Menariknya, hanya 1,3\% unit yang dikirim ke lokasi bencana dalam waktu 4 hari \citep{abolghasemi2008revisiting} . Meskipun banyak faktor yang dapat memperumit tanggap bencana, jelas bahwa distribusi, bukan pasokan, tetap menjadi masalah kritis.

Studi tentang kejadian serupa di Amerika Serikat memperkuat bahwa kekurangan produk darah pada saat bencana alam atau korban massal seringkali tidak menjadi masalah; sebaliknya, logistik distribusi adalah tantangannya. Satu studi menemukan bahwa hanya dalam 4 kasus dalam 25 tahun terakhir lebih dari 100 unit darah telah digunakan dalam 24 sampai 30 jam pertama setelah bencana di Amerika Serikat \citep{schmidt2002blood} \citep{abolghasemi2008revisiting}. Dalam tinjauan bencana baru-baru ini di Amerika Serikat di mana permohonan massal sering mengakibatkan peningkatan donor darah, penundaan yang signifikan ditemukan dalam distribusi sumbangan sensitif waktu ini \citep{klein2001earthquake}. Penting untuk dicatat bahwa, karena penyaringan dan pengujian laboratorium, darah biasanya tidak dapat digunakan pada tanggal disumbangkan. Namun demikian, kemampuan untuk secara cepat memindahkan produk darah antar pusat untuk mengatasi kekurangan, tanpa melibatkan manusia dalam proses pengangkutan, akan meningkatkan perawatan pasien dan mengurangi biaya.

Bank darah memiliki sistem pengamanan dan cadangan untuk mencegah kekurangan pada saat terjadi bencana atau peningkatan permintaan. Salah satu metode yang umum digunakan adalah dengan menyimpan sedikit suplai produk darah dan kemudian meminta darah, sesuai kebutuhan, dari bank darah daerah atau rumah sakit daerah. Meskipun sistem ini membantu mengurangi pemborosan produk darah, tingkat pemborosan yang dilaporkan masih berkisar antara 1\% hingga 26\% \citep{galloway2008tabletop}. Ketika terjadi peningkatan permintaan, produk darah kemudian dikirim melalui kurir, taksi, ambulans, atau kendaraan polisi \citep{sandler2002transportation}. 

Militer menggunakan metode yang lebih canggih, termasuk truk berpendingin, helikopter yang dipasang di bawahnya dengan beban selempang, dan parasut, untuk menyebarkan darah selama situasi pertempuran. Transportasi darat relatif murah, tetapi risiko terhadap personel tetap ada, dan transportasi dapat terhambat oleh cuaca, kondisi jalan, atau penghematan lingkungan. Transportasi udara konvensional dengan pesawat bersayap tetap atau bersayap putar mahal dan juga membahayakan awak. Meskipun beberapa jaringan trauma secara rutin mengirimkan produk darah dengan kru pengangkut helikopter, ini masih memerlukan pasien untuk kemudian diangkut ke pusat regional, yang menempatkan kru dan pasien pada risiko tambahan. Karena biayanya yang mahal, pesawat tidak secara rutin digunakan untuk mengangkut produk darah sendirian ke pasien.

Oleh karena itu, penggunaan UAV mungkin memiliki aplikasi di bidang kedokteran. Bukan hal yang aneh jika rumah sakit akses kritis memiliki persediaan obat yang terbatas dan variasi obat yang lebih sedikit dibandingkan dengan rumah sakit daerah. Antivenom, misalnya, jarang digunakan, memiliki masa simpan terbatas, dan mahal; Oleh karena itu, tidak praktis untuk menyimpannya di banyak rumah sakit. Akibatnya, pasien harus dipindahkan ke produk atau produk harus dikirim ke pasien, yang dapat menyebabkan penundaan perawatan yang signifikan. UAV dapat memenuhi peran pengirim tanpa risiko untuk mengangkut kru dan tanpa mengharuskan pasien untuk dipindahkan. Demikian pula, perangkat medis seperti perangkat fiksator eksternal, defibrilator otomatis, kasa tempur, dan torniket juga dapat dikirim oleh UAV jika diperlukan. Penggunaan UAV dalam bencana alam telah diusulkan oleh organisasi bantuan bencana. Namun, penggunaan ini juga dapat diperluas ke acara multi-korban di dalam negeri. Dalam keadaan ini, UAV berpotensi digunakan untuk mengangkut pasokan medis darurat ke rumah sakit setempat dan bahkan langsung ke pasien yang terluka di tempat kejadian.

Sistem pengiriman produk darah saat ini di Amerika Serikat bergantung pada kombinasi pemasok regional dan rumah sakit. Rumah sakit akses kritis terkecil di wilayah kami, biasanya 4 hingga 12 tempat tidur, secara rutin menyimpan 2 hingga 6 unit sel darah merah dalam inventarisnya dan tidak ada plasma beku segar atau trombosit. Rumah sakit dengan akses kritis yang lebih besar biasanya membawa 14 hingga 30 unit sel darah merah, 8 unit plasma, dan tidak ada trombosit atau kriopresipitat. Fasilitas ini tidak terlalu sering menggunakan produk darah, sehingga tidak jarang rumah sakit mengirimkan darah yang hampir kadaluarsa (dalam 7-10 hari kadaluarsa) kembali ke rumah sakit yang lebih besar untuk mencegah pemborosan. Rumah sakit berukuran sedang (20-50 tempat tidur) di wilayah kami membawa plasma minimal dan tidak ada trombosit. Hal ini menghasilkan suplai yang sangat terbatas yang mungkin tidak cukup untuk mendukung perdarahan yang signifikan atau protokol transfusi masif. Biasanya, hanya fasilitas regional yang memiliki 50 tempat tidur atau lebih yang memiliki suplai PRBC, plasma, trombosit, dan kriopresipitat yang ekstensif. Menyadari bahwa pedesaan Amerika dilayani oleh institusi medis yang lebih kecil ini, jelas bahwa kualitas perawatan dipengaruhi oleh biaya dan ketepatan waktu pengiriman produk darah, dan bahkan rumah sakit yang lebih besar dapat kehabisan jenis darah tertentu.