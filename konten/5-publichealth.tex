\section{Public Health dan Medical surveillance}
\label{sec:publichealth}

Drone digunakan untuk pengawasan lokasi bencana, area dengan bahaya biologis dan kimia, dan pelacakan lokasi penyebaran penyakit atau pandemi.
Telah ditunjukkan bahwa drone dapat mengumpulkan informasi tentang jumlah pasien yang membutuhkan perawatan dan triase di lingkungan berisiko tinggi. Pada tahun 2013, drone digunakan setelah Topan Haiyan di Filipina untuk memberikan pengawasan udara guna menilai kerusakan awal badai dan memprioritaskan upaya bantuan \citep{hlad2015drones}. Dalam upaya untuk meningkatkan efisiensi tim respons, Layanan Kesehatan Nasional di Inggris telah menyelidiki penggunaan drone untuk menilai cedera yang terkait dengan bahan kimia, biologi, dan nuklir \citep{rosser2018surgical}. 

Teknologi drone telah digunakan untuk mendeteksi bahaya kesehatan, seperti logam berat, aerosol, dan radiasi. Dalam sebuah penelitian dari Italia selatan, drone yang dilengkapi dengan perangkat lunak fotogrametri resolusi tinggi digunakan untuk mengakses dan memprediksi risiko kanker secara akurat dari konsentrasi tembaga tingkat tinggi di area pertanian \citep{capolupo2015photogrammetry}. \citet{brady2016characterization} mendemonstrasikan kemampuan drone quadrotor dengan platform pengambilan sampel bawaan untuk mengukur aerosol dan melacak level gas secara akurat di medan yang kompleks. Melalui deteksi dini, sistem ini dapat mencegah penyebaran bahaya kesehatan dari patogen. Sejalan dengan itu, teknologi drone juga telah digunakan untuk mendeteksi radionuklida yang khas dalam kecelakaan nuklir dan memetakan radiasi dari tambang uranium \citep{tang2016efficiency} \citep{martin2015use}. 

Selain itu, kemampuan drone untuk memperoleh informasi temporal dan spasial resolusi tinggi secara real-time dengan biaya rendah membuatnya layak untuk penelitian epidemiologi. Penggunaan tersebut melibatkan pemantauan deforestasi, perluasan pertanian, dan kegiatan lain yang mengubah habitat alami dan komunitas ekologis. \citet{fornace2014mapping}
mendemonstrasikan penggunaan drone untuk mengkarakterisasi perubahan lahan dan pola deforestasi di Malaysia yang mempengaruhi penyebaran zoonosis daripada parasit dari penyakit malaria. Dalam kasus penelitian lainnya, \citet{barasona2014unmanned} menggunakan drone untuk melacak distribusi spasial mamalia besar pembawa tuberkulosis di Spanyol selatan. Baru-baru ini, para peneliti telah menggunakan drone dengan modul analisis asam nukleat untuk mendeteksi Staphylococcus aureus dan virus Ebola \citep{priye2016lab}.

Salah satu penggunaan drone yang paling menjanjikan adalah di bidang telemedicine yang sedang berkembang—diagnosis jarak jauh dan perawatan pasien melalui teknologi telekomunikasi \citep{breen2010evolutionary}. Kata kunci dalam definisi telemedicine adalah telekomunikasi. Sayangnya, komunikasi yang diperlukan untuk misi telemedicine ke lingkungan terpencil, bantuan bencana, atau pertempuran tidak dapat bergantung pada jaringan komersial. Ide pendirian Infrastruktur Telekomunikasi Instan (ITI) menggunakan drone dibahas oleh penulis senior (JCR) di Athena, Yunani, pada tahun 1998 di Program Tele-medicine Pusat Ruang Angkasa Komersial Yale /NASA Gambar 4. Presentasi platform drone showcaseda yang berkonsentrasi pada penyediaan komunikasi untuk melakukan evaluasi pra dan pasca operasi pasien dan telementoring prosedur bedah tertentu di daerah terpencil. Telementoring adalah pemberian bimbingan jarak jauh oleh ahli bedah berpengalaman ataupun proseduralis ke rekan yang kurang berpengalaman, dengan prosedur yang muncul menggunakan komputer dan telekomunikasi \citep{rosser1997telementoring}. 

